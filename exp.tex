\documentclass{report}
\usepackage{hyperref}
\usepackage{fontspec,xunicode}
\usepackage{amsmath} 
\usepackage{graphicx}
\usepackage{nopageno}
\usepackage{ulem}
\setmainfont{AR PL UKai TW}
\usepackage{CJK}
\XeTeXlinebreaklocale "zh"
\usepackage{type1cm}
\begin{document}
\pagestyle{plain}
\fontsize{12pt}{20pt}\selectfont
\begin{center}
	\Huge{幹訓經傳文}\\ 
	\huge{Handsome Liu}\\ 
	\huge{2016.07.14}\\ 
\end{center}
	
  大家好,我是建國中學電子計算機研習社的學術瘦子,這次幹部訓練的執秘之一,也是學術課密碼學主講。這篇經驗傳承文主要是要告訴你們一個學術應該要做甚麼,還有如何把那些事情做好。\\
\\{\Large 教課}\\
  首先,作為一個學術,我們就有義務要教課。像我這種萬年副講可能沒啥資格說話,不過姑且還是說一下我對教課的理解好了。\\
  通常課程的主題是自行跟學術長討論,講師可能會有非學術組的人,但理論上是學術組的人優先當講師,我可能算是個例外吧。一週五天的社課當中,會有一天是聯合社課。通常學術長會先跟對面的學術長協調課程內容,能夠當上聯合社課的講師應該算是很幸運的事情,在友社打響知名度總不會是件壞事。\\
  教課的時候,最核心的要求是:「在一定的時間內,讓學員理解一定量的東西」。這可以分成兩個項目:「時間管理」和「促進學員理解」。\\
  時間管理的部分,如果教課的經驗不算太豐富,沒辦法直接從課程的內容推算出需要的時間的話,可能就要在事先備課的時候計一下時,時間太短就加內容;太長就砍內容。值得注意的是,每個人在群眾面前授課的時候,或多或少都會緊張,而緊張可能會讓說話速度變快或變慢,因人而異,通常是變快。那麼在計時的時候就要把這點考慮進去,此外也要記得預留讓學員發問的時間。\\
  而促進學員理解的方法,首要條理分明。讓自己說話有條理的方法有很多,其中一個是「知道自己現在在說什麼、以及等一下要說什麼」。這不僅能夠使說話的邏輯更容易被理解,也能讓自己口齒更加清晰。如此一來,事前的備課就更加重要了。\\
  除了口頭授課本身之外,也有很多方法來輔助學員理解。最常見的兩個方法,是製作投影片和編纂講義。\\
\\{\Large 投影片}\\
  製作投影片有很多好處,比方說能夠具現化一些口頭難以描述的圖片等內容,還有最重要的:有理由將螢幕切換成學員無法控制的模式,從而使學員更專心聽課。\\
  投影面的字不宜太小,行跟行之間也不宜太密,通常是以一頁 8 行為上限。\\
\\{\Large 講義}\\
  比起投影片,講義可以寫得更顯專業。可以寫得白話,但不需要像投影片那樣寫得搞笑或是偷放梗。我的密碼學講義應該算是寫得太嚴肅了,但那是我在兩天內趕出來的,原諒我吧。\\
\\{\Large 對內帶動學術風氣}\\
  建北電資就算活動辦得跟康輔社一樣,好歹也是個學術性社團。因此,學術組的人理所當然要會學術,而非學術組的就算不當學術主講,也得會一些背景知識。因為學術組的人可能不足五個,那麼五天的社課就勢必要找非學術組的來教。那麼,我們就必須帶動學術風氣,讓所有學弟妹都多少會一點學術。不管是教他們資訊、幫他們進資訊校隊培訓、拉他們去打校外資訊比賽都好。每個人擅長的領域都不同而重疊的部分很少,如此才能夠讓學弟妹學到最多。\\
\\{\Large 結語}\\
  這次是我第一次當講師,也是我第一次辦活動。或許這個幹部訓練也同時或多或少訓練到我了也說不定。希望我和你們都能在這五天的學術課中,學到很多東西,讓自己更加進步。也預祝各位在這次的幹部訓練中,能夠玩得開心,並且收穫滿滿的回家。\\
\end{document} 
