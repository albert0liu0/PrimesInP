\documentclass{article}
\usepackage[a4paper, total={6in, 8in}]{geometry}
\usepackage{hyperref}
\usepackage{calrsfs}
\usepackage{fontspec,xunicode}
\usepackage{fontenc}
\usepackage{titlesec}
\usepackage{amsmath} 
\usepackage{amsfonts} 
\usepackage{graphicx}
\usepackage{float}
\usepackage{xeCJK}
\usepackage{listings}
\usepackage[framemethod=TikZ]{mdframed}
\usepackage{mathtools}
\usepackage{indentfirst}
\usepackage{amssymb}
\DeclarePairedDelimiter\ceil{\lceil}{\rceil}
\DeclarePairedDelimiter\floor{\lfloor}{\rfloor}
\setCJKmainfont{AR PL UKai TW}
\lstset{basicstyle=\ttfamily,xleftmargin=24pt}
\titleformat{\section}[hang]{\normalfont\LARGE\bfseries}{\thesection}{0.5em}{}
\titleformat{\subsection}[hang]{\normalfont\large\bfseries}{ \thesubsection}{0.5em}{}
\setlength\parindent{24pt}
\setcounter{section}{-1}
\newcommand{\nequiv}{\not\equiv}
\DeclareMathAlphabet{\pazocal}{OMS}{zplm}{m}{n}
\newcommand{\Gc}{\pazocal{G}}
\setlength{\abovedisplayskip}{0pt}
\setlength{\belowdisplayskip}{0pt}
\setlength{\abovedisplayshortskip}{0pt}
\setlength{\belowdisplayshortskip}{0pt}
%\XeTeXlinebreaklocale "zh"
%\usepackage{type1cm}
\begin{document}
\fontsize{12pt}{20pt}\selectfont
\begin{center}
	\bfseries\huge{Cryptography Final Report\\PRIMES is in P}\\
\end{center}
{
    \hfill 臺灣大學 資訊工程學系 B04902012 劉瀚聲
}
\indent    在得知這學期的密碼學報告可以自選相關主題時,腦海中浮現的第一選擇,便是這篇「PRIMES is in P」。質數在大多數的密碼系統中,往往是相當重要的一環。身為一個資工系學生,目前的研究領域又是關於演算法與複雜度,這篇論文一直在我的 Wishing List 的前幾位。PRIMES屬於coNP相當顯然,但PRIMES在NP內並不直觀。可以想像,這篇論文的在2002年的發表震驚了多少猜測 PRIMES 屬於 NP-Complete 或 coNP-Complete 的學者。

    一些教授和網路上大多數的網友對這篇論文的評價都是「直白易懂」、「半小時內可以讀完」之類。但我親自讀之後,並不覺得它有傳聞中的那麼簡單。除了某些引理用到了一些我不曾學過的定義和性質(如分圓多項式等),相對影響較大的,是裡面某些推導或敘述並沒有給出詳細的原因,試著把敘述的正確性證明一次,卻發現原因並不顯然等諸如此類的情況。像這樣的心得,會以腳註的方式寫在報告中。

    為了證明自己有學到東西,這篇報告是用中文寫成,至少能表示有理解論文的內容而不是抄襲而得。此外,也把證明的架構改成自己比較喜歡的形式。我所讀過的論文,大多是 Top-down 的架構。在這篇報告中,定理會在後面的其他節證明,而引理會在當節的最後證明。如此一來,讀者在熟悉整個證明之前,便可以先了解證明的架構、知道每一個定理與引理在證明中的地位,之後再選擇想了解的定理與引理閱讀其證明。

    在開始寫起這篇報告後,才發現貫徹這些原則有許多困難:用中文寫數學證明實在是太痛苦了。一來是用英文思考再翻成中文往往很奇怪,二來是論文中有一些作者自己創造的專有名詞(如 "introspective",意譯成自省函數很奇怪,但原封不動的將英文單字夾在中文報告中又顯得突兀。幾經思索,決定按照函數的特性,擅自翻譯成冪同態)等等,果然還是好想用英文寫啊。

    正如我的專題指導教授所言,易寫則難讀,易讀則難寫。期許我的這篇報告費盡心思的結果,能讓所有想認識這個演算法的讀者快速了解其運作及正確性。

    整個演算法以定理2.1為核心。事實上,定理2.1已經提供了一個顯然充分的質數判別法:枚舉$[0,n-1]$中的所有$a$,並檢驗\\
    \centerline{$(X+a)^n\equiv X^n+a \pmod{n}$}\\
    是否成立,但這樣一來迭代的次數就高達$n$次,而且$n$次多項式的乘法與比較需要的時間也是$O(n)$,但輸入規模是$O(\lg n)$。在這個算法中,解決迭代次數過多的方法是將$a$的範圍縮減到$[0,l]$,而解決多項式次數過高的方法是在模$(X^r-1)$的剩餘類環上進行運算和比較。但如此一來,這個算法作為質數檢驗法的充分性就不再顯然,而$l$和$r$的上界也需要證明。

\section{Notation}
\section{Algorithm and Proof Sketch}
\begin{mdframed}
\noindent\textbf{Algorithm 1.1} 判定質數的算法
\begin{lstlisting}[mathescape=true]
0 Input: integer $n>1$.
1 If $n=a^b$ for some $a\in\mathbb{N}$ and $b>1$, output COMPOSITE.
2 Find the smallest $r$ such that $o_r(n)>\lg^2n$.
3 If $1<(a,n)<n$ for some $a\leq r$, output COMPOSITE.
4 If $n\leq r$, output PRIME.
5 For $a$ from $1$ to $\floor{\sqrt{\phi(r)}\lg n}$ do:
        If $(X+a)^n\nequiv X^n+a \pmod{X^r-1, n}$, output COMPOSITE.
6 Output PRIME.
\end{lstlisting}
\end{mdframed}

\begin{mdframed}
\noindent\textbf{Theorem 1.2} 當算法1.1輸出 \texttt{COMPOSITE}時,$n$為合數
\end{mdframed}

    第2節為定理1.2之證明。

\begin{mdframed}
\noindent\textbf{Theorem 1.3} 當算法1.1輸出 \texttt{PRIME}時,$n$為質數
\end{mdframed}

    第3節為定理1.3之證明。

\begin{mdframed}
\noindent\textbf{Theorem 1.4} 算法1.1的時間複雜度為 \(O(\lg^{12}n)\)
\end{mdframed}
    
    第4節為定理1.4之證明。

    定理1.2及定理1.3保證了算法的正確性。而由於輸入規模為$\lg n$,定理1.4保證了算法的運行時間為多項式時間。故算法1.1是一個PRIME的多項式時間算法。

\section{Correctness When Output COMPOSITE}

    當算法在第1步或第3步輸出COMPOSITE時,$n$是合數,因為第1步的$a$和第3步的$(a,n)$會是一個$n$的非平凡因數。

\begin{mdframed}
\noindent\textbf{Lemma 2.1} $a\in\mathbb{Z}$,$n\in\mathbb{N}$,$n\geq 2$,$(a,n)=1$,那麼$n$為質數若且唯若\\
    \centerline{$(X+a)^n\equiv X^n+a \pmod{n}$}
\end{mdframed}

    當算法在第5步輸出COMPOSITE時,表示$(X+a)^n\nequiv X^n+a \pmod{X^r-1, n}$,因此$(X+a)^n\nequiv X^n+a \pmod{n}$。根據引理2.1,$n$為合數。

\noindent\textit{2.1 Proof to Lemma 2.1} 

    根據二項式定理,$\forall 0<i<n$,$(X+a)^n$中$X^i$的係數為$C^n_ia^{n-i}$,而$X^n$的係數顯然同餘。

    \noindent (1) $n$為質數
    
        由於$\forall 0<i<n$,$C^n_i=\dfrac{n!}{i!(n-i)!}$,$n\mid(n!)$,$n\nmid(i!(n-i)!)$,故$C^n_ia^{n-i}\equiv 0\pmod{n}$。根據費瑪小定理,$a^n\equiv a\pmod{n}$。故$(X+a)^n\equiv X^n+a \pmod{n}$。

    \noindent (2) $n$為合數

        對於$n$的任何一個質因數$q$,若$q^k\mid n$但$q^{k+1}\nmid n$,那麼由於$C^n_q=\dfrac{n(n-1)...(n-q+1)}{q(q-1)(q-2)...(1)}$,而$q^k\nmid\dfrac{n}{q}$,且分子及分母中之其他項皆與$q$互質,故$n\nmid C^n_q$。亦即,$C^n_qa^{n-q}\not\equiv 0\pmod{n}$。

\section{Correctness When Output PRIME}

    當算法在第4步輸出PRIME時,$n$為質數。因為第3步和$n\leq r$保證了$\forall a<n$,$(a,n)=1$。以下證明算法在第6步輸出PRIME的正確性。

    由於$o_r(n)>1$,$n$必定有質因數$p$滿足$o_r(p)>1$。又因為通過了第3步和第4步的檢驗,所以$(r,n)=1$,且$p>r$,故$p,n\in \mathbb{Z}_r^*$。另外,令$l=\floor{\sqrt{\phi(r)}\lg n}$。在第5步中,算法檢驗了$l$個等式。由於第5步沒有輸出COMPOSITE,因此對於所有的$0\leq a\leq l$,都有:\\
    \centerline{$(X+a)^n\equiv X^n+a\pmod{X^r-1, n}$}\\
由於$p$是$n$的因數,故:\\
    \centerline{$(X+a)^n\equiv X^n+a\pmod{X^r-1, n}$}\\

\begin{mdframed}
\noindent\textbf{Lemma 3.1} $(X+a)^{\frac{n}{p}}\equiv X^{\frac{n}{p}}+a\pmod{X^r-1, p}$。
\end{mdframed}

    對於多項式函數$f$和自然數$m$,定義$m$對於$f$是冪同態的,表示$f(X)^m\equiv f(X^m)\pmod{X^r-1, p}$。由上面的敘述可知,對所有的$0\leq a\leq l$,$n$、$p$、$\dfrac{n}{p}$對於$(X+a)$都是冪同態的。

\begin{mdframed}
\noindent\textbf{Lemma 3.2} 如果$m_1$、$m_2$對於$f(X)$都是冪同態的,那麼$m_1m_2$對於$f(X)$也是冪同態的。
\end{mdframed}

\begin{mdframed}
\noindent\textbf{Lemma 3.3} 如果$m$對於$f_1(X)$、$f_2(X)$都是冪同態的,那麼$m$對於$f_1(X)f_2(X)$也是冪同態的。
\end{mdframed}

    接下來考慮以下幾個集合:令$I=\{p^i(\dfrac{n}{p})^j|i, j\geq 0\}$,$P=\{\prod\limits_{a=0}^{l}(X+a)^{e_a}|e_a\geq 0\}$。根據引理3.2和3.3,$I$中的每一個元素對於$P$中的每一個元素都是冪同態的。由於$n$、$p$、$\dfrac{n}{p}$都分別和$r$互質,故$I$模$r$所形成的集合會在$\mathbb{Z}_r^*$內,而且是一個群。令$G$為那個群,亦即,$G=I/r\mathbb{Z}=\{p^i(\dfrac{n}{p})^j \mod r|i, j\geq 0\}$,並令$t=|G|$。由於$o_r(n)>\lg^2n$,故$t>\lg^2n$。令$Q_r(X)$為$r$次分圓多項式,根據分圓多項式的性質,$Q_r(X)\mid (X^r-1)$,且在$\mathbb{F}_p$上,$Q_r(X)$是若干個$o_r(p)$次不可約多項式的乘積。令$h(X)$是其中一個這樣的多項式。由於$o_r(p)>1$,故$deg(h(X))>1$。令$\Gc$為$P$模$h(X)$再把係數模$p$得到的集合。由於$h(X)$不可約,故$\Gc$是一個乘法群。$\Gc$可以看作是由$X$、$(X+1)$、$(X+2)$、......、$(X+l)$所生成的、在$\mathbb{F}_p/h(X)$上的乘法群。

\begin{mdframed}
\noindent\textbf{Lemma 3.4} $|\Gc|\geq\binom{t+l}{t-1}$。
\end{mdframed}

\begin{mdframed}
\noindent\textbf{Lemma 3.5} 若$n$不是$p$的冪次,則$|\Gc|\leq n^{\sqrt{t}}$。
\end{mdframed}

    根據引理3.4,
        \begin{align*}
        |\Gc| &\geq \binom{t+l}{t-1}\\
              &\geq \binom{l+1+\floor{\sqrt{t}\lg n}}{\floor{\sqrt{t}\lg n}}\\
              &\geq \binom{2\floor{\sqrt{t}\lg n}+1}{\floor{\sqrt{t}\lg n}}\\
              &> 2^{\floor{\sqrt{t}\lg n}+1}\\
              &\geq n^{\sqrt{t}}
        \end{align*}
    故根據引理3.5,$n$為$p$的冪次。但由於第1步沒有輸出COMPOSITE,故$n=p$,即$n$是個質數,定理得證。

\noindent\textit{3.1 Proof to lemma 3.1}
    \begin{align*}
    (X^{\frac{n}{p}}+a)^p&\equiv X^n+a\\
                         &\equiv ((X+a)^{\frac{n}{p}})^p\pmod{X^r-1,p}
    \end{align*}

    故只須證明對於多項式$f(X)$、$g(X)\in \mathbb{Z}_{p}[X]/(X^r-1)$,$f^p(X)\equiv g^p(X)\pmod{X^r-1,p}\Rightarrow f(X)\equiv g(X)\pmod{X^r-1,p}$。

    但$f^p(X)-g^p(X)\equiv 0\pmod{X^r-1,p}\Rightarrow (f(X)-g(X))^p\equiv 0\pmod{X^r-1,p}$(證明類似定理2.1的證明,展開後觀察係數)。故進一步,只須證明$(f(X)-g(X))^p\equiv 0\pmod{X^r-1,p}\Rightarrow(f(X)-g(X))\equiv 0\pmod{X^r-1,p}$。其充分條件是對於多項式$z(X)\in\mathbb{Z}_{p}[X]/(X^r-1)$,$z^p(X)\equiv 0\pmod{X^r-1,p}\Rightarrow z(X)\equiv 0\pmod{X^r-1,p}$。

    假設$z^p(X)\equiv 0\pmod{X^r-1,p}$,那麼存在多項式$q(X)$使得$z^p(X)\equiv q(X)(X^r-1)\pmod{p}$。若$(X^r-1)\nmid z(X)$,那麼$(X^r-1)$必有重數大於$1$的因式。亦即存在多項式$q_1(X)$、$q_2(X)$滿足$(X^r-1)\equiv [q_1(X)]^2q_2(X)\pmod{p}$,且$\deg(q_1(X))>1$。但考慮$(X^r-1)$之形式微分:$(X^r-1)'\equiv rX^{r-1}\pmod{p}$,$([q_1(X)]^2q_2(X))'\equiv 2q_1(X)q_1'(X)q_2(X)+[q_1(X)^2q_2'(X)\pmod{p}$。但$(X^r-1,(X^r-1)')\equiv 1\pmod{p}$,$q_1(X)\mid([q_1(X)]^2q_2(X), ([q_1(X)]^2q_2(X))')$,矛盾,故$(X^r-1)\mid z(X)$,即引理得證。

\noindent\textit{3.2 Proof to lemma 3.2}
    \begin{align*}
    [f(X)]^{m_1m_2}&\equiv [f(X^{m_1})]^{m_2}\\
    \end{align*}

\noindent\textit{3.3 Proof to lemma 3.3}

\noindent\textit{3.4 Proof to lemma 3.4}

\noindent\textit{3.5 Proof to lemma 3.5}

\section{Time Complexity of Algorithm}

    第1步枚舉$b$自$1$至$\lg n$,二分搜尋對應的$a$,檢驗$a^b$與$n$的關係。時間複雜度$O(\lg^3n)$。

    第2步自1開始枚舉$r$,檢驗是否$\forall 1\leq i\leq\lg^2n$,$n^i\nequiv 1\pmod{r}$。時間複雜度$O(r\lg n)$。

    第3步枚舉$a$自$1$至$r$,計算$(a,n)$。時間複雜度$O(r\lg n)$。

    第4步時間複雜度$O(1)$。

    第5步每次迭代可用快速冪在$\lg n$次多項式乘法內算出$(X+a)^n\pmod{X^r-1, n}$的值,而多項式的次數不超過$r$,故時間複雜度為$O((\sqrt{\phi(r)}\lg n)(r^2\lg n))=O(r^{\frac{5}{2}}\lg^2n)$。

    第6步時間複雜度$O(1)$。

    故整體時間瓶頸為第5步,複雜度為$O(r^{\frac{5}{2}}\lg^2n)$。故只須證明$r$的上界。

    令$B=\ceil{\lg^5n}$,$S=n^{\floor{\lg B}}\prod\limits_{i=1}^{\floor{\lg^2n}}(n^i-1)$。

    $S<n^{\floor{\lg B}}\prod\limits_{i=1}^{\floor{\lg^2n}}(n^i)=n^{\floor{\lg B}+\frac{1}{2}\lg^2n(\lg^2n-1)}\leq n^{\lg^4n}\leq 2^B$。

    考慮$R=\min\{R'|R'\nmid S\}$,由於$\forall i\in[1, \floor{\lg^2n}], R\nmid(n^i-1)$,故$o_R(n)>\lg^2n$,亦即$R$是$r$的一個上界。

\begin{mdframed}
\noindent\textbf{Lemma 4.1} \textit{(Nair [1])} 令$LCM(m)$表示前$m$個自然數的最小公倍數,那麼對於$m\geq 7$,$LCM(m)\geq 2^m$。
\end{mdframed}

    假設$R>B$,那麼$\forall i\leq B$,$i\mid S$,亦即$S\mid LCM(B)$,但根據引理4.1,$LCM(B)\geq 2^B>S$,矛盾,故$R\leq B$。從而,$r\leq \lg^5n$,定理得證。

\section{References}

    [1] M. Nair. On Chebyshev-type inequalities for primes. \textit{Amer. Math. Monthly} 89:126 129, 1982.
    
\end{document} 
